\addcontentsline{toc}{chapter}{Avalia\c{c}\~ao}
\chapter*{Avalia\c{c}\~ao}

Teremos 3 tipos de atividades avaliativas:

\begin{itemize}
\item \textbf{Prova}: Avalia\c{c}\~ao individual, escrita e sem consulta.
\item \textbf{Desafio}: Avali\c{c}\~ao em grupo, escrita e com consulta.
\item \textbf{Projeto de Programa}: Trabalho pr\'atico de programa\c{c}\~ao em grupo.
\end{itemize}

Os pesos de cada atividade avaliativa est\~ao definidos na Tabela \ref{tab:avaliacao}

\begin{table}[htdp]
\caption{Atividades Avaliativas}
\begin{center}
\begin{tabular}{lr}
\hline
\textbf{Atividade} & \textbf{Peso} \\
\hline
Prova 1 & 30\% \\
\hline
Prova 2 & 30\% \\
\hline
Desafios & 30\% \\
\hline
Projetos de Programa\c{c}\~ao & 10\% \\
\hline
\end{tabular}
\end{center}
\label{tab:avaliacao}
\end{table}%

A nota final ($NF$) no curso ser\'a calculada com a seguir:
\begin{equation}
NF = NP1 * 0,3 +  NP2 * 0,3 +  MD * 0,3 + MP * 0,1,
\end{equation}

onde $NP1$ \'e a nota na Prova 1, $NP2$ \'e a nota na Prova 2, $MD$ \'e a m\'edia aritm\'etica das notas nos desafios e $MP$ \'e a m\'edia aritm\'etica das notas nos projetos de programa\c{c}\~ao.

Para ser aprovado, o estudante precisa obter $NF \geq 60$.
