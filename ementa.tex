%----------------------------------------------------------
\addcontentsline{toc}{chapter}{Ementa}
\chapter*{Ementa}

Os t\'opicos estudados neste curso est\~ao divididos em seis partes.

neumar

\begin{enumerate}
\item L\'ogica e Demonstra\c{c}\~oes
\begin{enumerate}
\item L\'ogica Proposicional
\item Predicados e quantificadores
\item M\'etodos (t\'ecnicas) de demonstra\c{c}\~ao
\end{enumerate}

\item Teoria dos N\'umeros
\begin{enumerate}
\item Sequ\^encias e somat\'orios (conjuntos enumer\'aveis)
\item Os n\'umeros inteiros e a divis\~ao
\item Os n\'umeros primos e os M\'aximos divisores comuns
\end{enumerate}

\item Indu\c{c}\~ao e Recursividade
\begin{enumerate}
\item Indu\c{c}\~ao Matem\'atica
\item Indu\c{c}\~ao completa e boa ordena\c{c}\~ao
\item Defini\c{c}\~oes recursivas e indu\c{c}\~ao estrutural
\item Rela\c{c}\~oes de recorr\^encia
\end{enumerate}

\item Contagem
\begin{enumerate}
\item Conceitos b\'asicos de contagem
\item Princ\'ipio da casa dos pombos
\item Permuta\c{c}\~oes e combina\c{c}\~oes
\item Princ\'ipio da inclus\~ao-exclus\~ao
\end{enumerate}

\item Rela\c{c}\~oes
\begin{enumerate}
\item Fun\c{c}\~oes
\item Rela\c{c}\~oes e suas propriedades
\item Fecho de rela\c{c}\~oes
\item Rela\c{c}\~oes de equival\^encia
\item Ordens parciais
\end{enumerate}

\item Grafos
\begin{enumerate}
\item Conceitos b\'asicos em Grafos
\item Representa\c{c}\~ao e Isomorfismo
\item Concectividade
\item Caminhos eulerianos e hamiltonianos
\item \'Arvores
\end{enumerate}
\end{enumerate}
